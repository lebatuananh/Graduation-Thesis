\chapter*{Tóm tắt báo cáo}
\addcontentsline{toc}{chapter}{Tóm tắt báo cáo}
Thương mại điện tử là toàn bộ chu trình và các hoạt động kinh doanh liên quan đến các tổ chức hay cá nhân tiến hành hoạt động thương mại có sử dụng các phương tiện điện tử và công nghệ xử lý thông tin số hoá, bao gồm cả sản xuất, phân phối, marketing, mua – bán, giao hàng hoá và dịch vụ bằng các phương tiện điện tử.
\par
Xuất phát từ những yêu cầu đó, đồ án này tập trung vào việc xây dựng và phát triển website thương mại điện tử để phục vụ nhu cầu về các hệ thống thương mại điện tử ở Việt Nam. Website thương mại điện tử là 1 website cung cấp cho người sử dụng và các nhà cung cấp, phân phối, quản lý hàng hoá quản lý những hoạt động của 1 cửa hàng hay 1 hệ thống cửa hàng vừa và nhỏ. Đối với khách hàng website cũng cấp những chức năng tìm kiếm, xem thông tin sản phẩm, xem địa chỉ và thông tin người bán hàng, cho phép người mua hàng có thể đặt hàng trực tuyến qua nhiều phương thức: giao dịch qua Ngân Lượng, đặt hàng COD, giao dịch thông qua ATM. Đối với các nhà quản lý tương tác với website theo các cấp độ người dùng, những nhà quản lý có thể thêm, sửa, xoá sản phẩm của mình theo quyền được phân từ người quản trị, họ cũng có thể quản lý các đơn hàng hay xem doanh thu của cửa hàng mình qua biểu đồ.
\par 
Qua đó, đề tài của em sẽ được chia ra thành các mục sau:
\begin{itemize}
\item CHƯƠNG 1: Tổng quan đề tài.
\par
Trong chương tổng quan đề tài, em nêu ra những mục chính là: Lý do chọn đề tài, mục tiêu của đề tài, các phương pháp sử dụng trong đồ án và cuối cùng là kết luận.
\item CHƯƠNG 2: Cơ sở lý thuyết
\par
Để xây dựng và hệ thống website thương mại điện tử, em trình bày một số cơ sở lý thuyết làm nền tảng để đáp ứng yêu cầu kỹ thuật đề xuất như là: 
\begin{itemize}
\item Back-end: ASP.NET Core, Entity Framework Core, Identity Framework Core.
\item Front-end: HTML (HyperText Markup Language), CSS (Cascading Style Sheets), Bootstrap, JS (Java Script), Jquery.
\end{itemize}
\item CHƯƠNG 3: Phân tích, thiết kế hệ thống.\par
Chương này đưa ra những phân tích yêu cầu và thiết kế chi tiết cho website thương mại điện tử.

\item CHƯƠNG 4: Triển khai, kiểm thử và kết quả.\par
Sau khi đưa ra các phân tích thiết kế ở chương 3 em tiến hành cài đặt kiểm thử các chức năng đã đề ra. Các kết quả triển khai kiểm thử phần mềm.

\end{itemize}

