\chapter{Phân tích thiết kế hệ thống}
\label{chapter3}
Từ những cơ sở lý thuyết thu thập ở chương \ref{chapter2}, chương \ref{chapter3} em sẽ trình bày phân tích yêu cầu chức năng quy trình thiết kế hệ thống.
\section{Phân tích yêu cầu}
\subsection{Mô tả hệ thống}
Một website thương mại cho phép cửa hàng có thể bán tất cả các loại mặt hàng, có thể giao dịch trực tuyển qua nhiều hinhf thức thanh toán như là: Ngân Lượng, Bảo Kim, ATM, COD,...
Chủ đạo của website là đăng thông tin sản phẩm thật cho khách hàng dễ dàng tìm kiểm thông qua mạng Internet rất phổ cập hiện nay. Người quản trị chính là chủ cửa hàng có một trang quản trị riêng đẻ đăng những thông tin sản phầm, quản lý hoá đơn mua hàng, xem doanh thu theo những mốc thời gian mà người quản trị cần, quản lý thông tin cửa hàng, quản lý quyền cho hệ thống cũng như những tài khoản của khách hàng và cho phép cửa hàng có thể bán thêm những mặt hàng mới nên sẽ phải có hệ thống danh mục sản phầm động. Để thuận tiện cho người quản lý phải có thêm chức năng tìm kiếm ở mỗi chức năng quản lý.
\par
Đối với khách hàng họ cần có 1 tài khoản với thông tin chính xác đề người quản trị liên hệ lại với họ và tiền hành giao dịch mua bán. Với trang chủ để khách hàng xem thông tin sản phẩm chúng ta cần phải có 1 giao diện bắt mắt và thân thiện với người dùng kèm theo đấy là những chức năng tìm kiếm theo những điều kiện khác nhau để người dùng có thể xem những sản phầm họ cần thật nhanh. Vì đây là 1 trang web thương mại điện tử chúng ta không thể thiếu đi được chức năng đăng ký đăng nhập cũng như thay đổi các thông tin cá nhân liên quan đến tài khoản người dùng. Một điều quan trọng hơn nữa là phải lưu được hoá đơn kèm theo đối với mỗi tài khoản người dùng và hỗ trợ chức năng thêm nhiều sản phẩm vào giỏ hàng để người sử dụng dễ dàng hơn trong việc mua sắm.
\subsection{Yêu cầu hệ thống}
\subsubsection{Yêu cầu chức năng}
 Hệ thống website thương mại điện tử sẽ được chia theo nhiều cấp độ người dùng:
\begin{itemize}
 \item Người quản trị: Có mọi quyền để tương tác trong hệ thống, người quản trị có thể thay đối bất kì thông tin của sản phẩm hay đăng bài viết của cửa hàng mình lên tràng chủ, người quản lý có thêm thêm nhân viên giao hàng vào các đơn hàng mà phương thức vận chuyển là COD, người quản lý có thể xem doanh thu bán hàng.  
 \item Nhân viên quản lý của hàng: Chỉ có thế xem và sửa thông tin các sản phầm để cập nhật nhanh chóng lên trang chủ phục vụ nhu cầu của khách hàng.
 \item Người vận chuyển hàng: Chỉ được xem hoá đơn theo trạng thái COD và vận chuyển hàng theo đơn hàng đã được giao trong hệ thống.
 \item Khách hàng: Khách hàng chỉ được phép truy cập ở trang chủ còn trang quản trị thì họ không có quyền gì, khách hàng có thể đăng kí, đăng nhập, và thay đổi thông tin tài khoản để mua hàng, họ cũng có thể gửi những đóng góp cho người quản trị để phát triền website hơn nữa.
\end{itemize}  
\subsubsection{Yêu cầu phi chức năng}
Ngoài các yêu cầu chức năng hệ thống còn có những yêu cầu phi chức năng sau:
\begin{itemize}
\item Hệ thồng có dung lượng không quá lớn, tốc độ xử lý nhanh hơn.
\item Công việc thực hiện chính xác.
\item Sử dụng mã hoá các thông tin nhạy cảm của khách hàng như: mật khẩu, ngày sinh nhật, số điện thoại,...
\item Đảm bảo an toàn dữ liệu khi chạy website trực tuyến
\end{itemize}
\section{Thiết kế hệ thống}
\subsection{Biểu đồ Use case diagram}
Một biểu đồ Use case chỉ ra một số lượng các tác nhân ngoại cảnh và mối liên
kết của chúng đối với Use case mà hệ thống cung cấp. Một Use case là một lời miêu tả của một chức năng mà hệ thống cung cấp. Lời miêu tả Use case thường là một văn bản tài liệu, nhưng kèm theo đó cũng có thể là một biểu đồ hoạt động. Các Use case được miêu tả duy nhất theo hướng nhìn từ ngoài vào của các tác nhân (hành vi của hệ thống theo như sự mong đợi của người sử dụng), không miêu tả chức năng được cung cấp sẽ hoạt động nội bộ bên trong hệ thống ra sao. Các Use case định nghĩa các yêu cầu về mặt chức năng đối với hệ thống.
\par




