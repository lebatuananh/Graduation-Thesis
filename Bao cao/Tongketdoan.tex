\thispagestyle{plain}
\chapter*{Kết luận}
\addcontentsline{toc}{chapter}{Kết luận}
Sau một thời gian làm đồ án, em nhận thấy bản thân đã có sự tiến bộ về khả nâng lập trình ứng dụng website, cải thiện tư duy về các sơ đồ chức năng mà nó gắn liền với các thuật toán, khả năng đọc tài liệu trên internet cũng như cách tìm từ khoá trên Google để giải quyết vấn đề, nắm rõ được các công nghệ sử dụng có trong hệ thống. Việc thực hiện đồ án cũng giúp em có một sản phẩm để đi tuyển dụng đồng thời làm em hứng thú hơn với nghề lập trình viên. Sau đây em xin trình bày một số yêu cầu đã thực hiện được cùng với phương hướng phát triển đề tài
\par
- Hoàn thành hệ thống website thương mại điện tử với các chức năng:
\begin{itemize}
	\item Người quản trị quản lý sản phẩm, danh mục sản phẩm, hoá đơn bán hàng, doanh thu bán hàng, nhóm quyền, tài khoản người dùng, các thông tin liên quan đến trang web: quảng cáo, slide ảnh, thông tin cửa hàng,...
	\item Người dùng xem sản phẩm dễ dàng với chức năng tìm kiếm theo gợi ý, và theo sắp xếp các thuộc tính sản phẩm. Họ cũng có thể đặt hàng trên trang chủ cùng với đấy họ có thể thanh toán trực tuyến thông qua Ngân Lượng.
\end{itemize}

- Hướng phát triển của đề tài:
\begin{itemize}
\item Chỉnh sửa giao diện người dùng bắt mắt.
\item Phát triển việc dùng cơ sở dữ liệu bằng điện toán đám mây để tăng tốc độ tải dữ liệu.
\item Tối ưu hoá tốc độ truy vấn dữ liệu bằng ngôn ngữ NoSQL.
\end{itemize}
\par 
Trong quá trình nghiên cứu, xây dựng và thiết kế sản phẩm, em đã vấp phải nhiều khó khăn. Nhưng nhờ có sự hướng dẫn tận tình của TS. Trần Quang Vinh cũng như sự giúp đỡ của các thành viên trong phòng nghiên cứu Sanslab đã giúp em hoàn thiện đồ án này. Qua đó, giúp em củng cố thêm những kiến thức về website, lập trình; những kỹ năng làm việc nhóm, tìm hiểu nghiên cứu những vấn đề mới; cũng như có được những kiến thức mới và trải nghiệm thực tế.