\chapter{Đặt vấn đề}
\label{chapter1}
Trong chương \ref{chapter1}, em nêu ra những mục chính là: Lý do chọn đề tài, mục tiêu của đề tài, các phương pháp sử dụng trong đồ án và cuối cùng là kết luận.
\section{Tổng quan về đề tài}
Thương mại điện tử là một khái niệm mới. Mặc dù ra đời chưa lâu nhưng nó đã nhanh chóng khẳng định được vị thế của mình nhờ sức hấp dẫn cũng như trên đà khá ngoạn ngục. Cùng với sự phát triển chóng mặt của Internet, thương mại điện tử đang có những bước tiến rất nhanh với tốc độ ngày càng cao.
\par
Cuối những năm 1900, thương mại điện tử vẫn còn là một khái niệm khá mới mẻ ở Việt Nam. Nhưng dưới sức lan toả rộng khắp của thương mại điện tử, các công ty Việt Nam cũng đang từng bước làm quen với phương thức kinh doanh hiện đại này \textls[-5]{Một số dự án thương mại điện tử ở Việt Nam đang phát triền rất mạnh mẽ như:}
\begin{itemize}
\item	thegioididong.vn, chuỗi cửa hàng thegioididong.com được thành lập từ 2004 chuyên bán lẻ các sản phẩm kỹ thuật số di động bao gồm điện thoại di động, máy tính bảng, laptop và phụ kiện điện tử với gần 1000 siêu thị tại 64 tỉnh thành trên khắp Việt Nam. Thế giới di động đã xây dựng được một dịch vụ khách hàng khác biệt vượt trội với văn hoá “Đặt khách hàng làm trung tâm” trong mọi suy nghĩ và hành động của mình với tất cả các nhân viên của công ty. Với uy tín của mình, thế giới di động hiện nay đứng top 1 bán hàng trực tuyến uy tín chất lượng với các mặt hàng di động, máy tính xách tay các sản phẩm công nghệ cao.
\item	lazada.vn, là một website mua sắm trực tuyến được thiết kế chuyên nghiệp, hiện đại. Lazada là trang web được nhiều người sử dụng nhất hiện nay tại Việt Nam. Tuy nhiên công ty chủ quản của Lazada không phải của Việt Nam mà là của công ty Singapore có chi nhánh tại Việt Nam. Lazada là công ty bán hàng trực tuyến lớn nhất tại khu vực Đông Nam Á với rất nhiều chi nhánh tại các nước lớn khu vực như Indonesia, Thailand, Philippines, Malaysia, Singapore. Tuy nhiên Lazada không cung cấp tất cả các hàng hóa dịch vụ mà chủ yếu là tạo ra sàn giao dịch online cho các cửa hàng đăng ký bán hàng trên website, công ty chỉ đảm bảo về giao dịch trực tuyến và quản lý cửa hàng, khách hàng.
\item	tiki.vn, Tiki là một trong những trang web mua sắm trực tuyến hàng đầu Việt Nam sở hữu hơn 800.000 khách hàng và cung cấp đến 120.000 sản phẩm thuộc 10 ngành hàng khác nhau như: Sách, Làm đẹp – Sức khoẻ, Nhà cửa – Đời sống, Điện thoại – Máy tính bảng, Thiết bị số – Phụ kiện số, Điện gia dụng, Thiết bị văn phòng phẩm, Mẹ và Bé, Đồ chơi – Đồ lưu niệm, Thể thao – Dã ngoại với mức doanh số tăng trưởng gấp ba lần mỗi năm. Tiki.vn đã được trao tặng danh hiệu “website TMĐT được yêu thích năm 2014” do người tiêu dùng bình chọn sau 5 năm nỗ lực hoạt động không ngừng nghỉ. Mạng lưới giao hàng của TiKi phục vụ trên toàn quốc, miễn phí cho mọi đơn hàng từ 250.000đ, riêng tại TPHCM và Hà Nội chỉ từ 150.000đ. Dịch vụ vận chuyển trong 24h giúp khách hàng trải nghiệm mua sắm trực tuyến một cách tiện lợi vừa tiết kiệm được thời gian, công sức mà vẫn bảo đảm được các quyền lợi về bảo hành hay đổi/trả dễ dàng trong vòng 30 ngày.
\item sendo.vn, Là trang web mua bán trực tuyến của Tập đoàn FPT nhằm kết nối người mua và người bán trên toàn quốc. Ra đời là một dự án Thương mại Điện tử do Công ty CP Dịch vụ Trực tuyến FPT (FPT Online) xây dựng và phát triển, Sendo.vn chính thức ra mắt người dùng vào tháng 9/2012. Ngày 13/5/2014, Công ty CP Công nghệ Sen Đỏ được thành lập, trực thuộc Tập đoàn FPT, là công ty chủ quản của Sendo.vn.
\item robins.vn Zalora là trang web mua sắm thời trang quốc tế được đầu tư vào thị trường Việt Nam khá mạnh trong thời gian gân đây. Với sự uy tín cao và nguồn vốn đầu tư khá mạnh, Zalora không khó để chiếm lĩnh thị trường thời trang online tại Việt Nam. Zalora có một hệ thống mua hàng trực tuyến nhanh gọn, dễ dàng, thuận tiện và tiết kiệm thời gian cho khách hàng. Ở Zalora bạn có thể mua tất cả món hàng thời trang nào mà bạn cần.
\item vatgia.com, Khi truy cập chợ mua bán online www.vatgia.com, khách hàng dễ dàng tìm kiếm thông tin với hàng nghìn gian hàng và sản phẩm về điện tử, công nghiệp,xe cộ, xây dựng – nhà cửa,hay các dịch vụ và giải trí. Gian hàng được trình bày một cách khoa học, kết hợp với nhiều công cụ tìm kiếm tiện lợi, dễ dàng cho người tiêu dùng có thể tìm được sản phẩm như mong muốn, rẻ, chính xác và trong thời gian ngắn nhất. Bên cạnh đó, website thương mại điện tử này còn cung cấp cho người tiêu dùng công cụ bình chọn đánh giá chất lượng dịch vụ của người bán, không gian để nhiều người đóng góp ý kiến, trao đổi thông tin về sản phẩm để tìm được những sản phẩm dịch vụ có giá cả và chất lượng tốt nhất.
\item chotot.com, Chợ Tốt ra đời vào năm 2012, là một kênh rao vặt trung gian, kết nối người bán và người mua bằng những giao dịch đơn giản, tiện lợi, nhanh chóng, an toàn. Tại Chợ Tốt, người dùng dễ dàng mua bán, mọi mặt hàng, dù đó là đồ cũ hay đồ mới. Các lĩnh vực như bất động sản, xe cộ, đồ dùng cá nhân, đồ điện tử,...\cite{1}
\end{itemize}
\par
Trong cuộc cách mạng công nghệ 4.0 các công nghệ thay nhau áp dụng dụng vào thương mại để phục vụ đời sống hàng ngày cho con người, thương mại điện tử giúp cho các nhà quản lý một phần nào trong việc mang sản phầm của mình đến với người sử dụng, cũng giúp cho người tiêu dùng không mất quá nhiều thời gian để tiêu dùng.
\section{Mục tiêu của đề tài}
Trên thực tế các cửa hàng vẫn bán hàng theo phương thức truyền thống ngưới mua hàng phải đến tận nơi để xem hàng hoá và mua, vào thời kì cách mạng công nghệ 4.0 chúng ta phải đưa công nghệ vào thương mại để xoá bỏ phương thức mua bán truyền thông. Dựa vào các lý do trên nên em đưa ra yêu cầu đối với website thương mại điện tử như sau:
\begin{itemize}
\item Đối với người quản lý:
\begin{itemize}
\item Phân quyền theo từng chức năng đối với các cấp người sử dụng website.
\item Cho phép người quản trị có tất các quyền đối với hệ thống.
\item Quản lý sản phẩm theo các thuộc tính của sản phẩm: ảnh, số lượng, giá bán, màu sắc, giá bán sỉ, kích cỡ,...
\item Quản lý các danh mục sản phầm để người quản lý có thể mở rộng mặt hàng.
\item Quản lý người vận chuyển hàng.
\item Quản lý các đơn hàng.
\item Xem doanh thu theo tháng, hay các chuỗi ngày.
\item Quản lý các quảng cáo, ảnh giao diện, thông tin khách hàng, địa chỉ liên hệ với cửa hàng.
\item Quản lý bài đăng cho website.
\end{itemize}
\item Đối với người tiêu dùng:
\begin{itemize}
\item Giao diện người dùng thân thiện với người dùng.
\item Hiển thị các danh mục sản phẩm để người dùng dễ chọn lựa.
\item Tìm kiếm theo gợi ý theo tên sản phẩm có trong hệ thống.
\item Hiển thị thông tin sản phầm rõ ràng và đầy đủ.
\item Giao diện thanh toán dễ dàng cho người mua hàng.
\item Đăng ký, đăng nhập cho người dùng để lưu thông tin vào đơn hàng để người quản lý dễ dàng giao hàng đến người tiêu dùng.
\item Chức năng quên mật khẩu để người dùng có thể lấy lại tài khoản để xem những đơn hàng mà người dùng đã mua ở website.
\end{itemize}
\end{itemize}
\section{Các phương pháp sử dụng trong thiết kế}
Các phương pháp mà em sử dụng gồm có:
\begin{itemize}
\item   Tham khảo tài liệu: tham khảo tài liệu từ sách báo về điện tử, từ mạng internet, kế thừa và phát triển các kết quả nghiên cứu đã có.
\item   Quan sát, học hỏi: thảo luận cùng thầy và các bạn để đưa ra được hướng đi tốt nhất, đạt hiệu quả cao,
\item   Thực hành và sửa lỗi: tiến hành viết code và sửa lỗi để đạt được kết quả tối ưu nhất,
\end{itemize}\par
\section{Kết luận}
Mục tiêu của đề tài này là xây dựng website thương mại điện tử để thay thế phương thức mua bán truyền thông. Để hoàn thành đề tài cần có  những kiến thức về phân tích thiết kế hướng đối tượng, quy trình kiếm thử phần mềm, quy trình thiết kế phần mềm, kiến thức về lập trình hướng đối tượng, và những kiến thức thực tế về thương mại điện tử em tìm hiểu được từ các trang web thực tế. Từ đây em có những kế hoạch cụ thể để hoàn thành mục tiêu đề ra và được em trình bày ở các chương sau.

