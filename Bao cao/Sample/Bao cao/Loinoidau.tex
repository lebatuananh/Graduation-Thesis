\thispagestyle{plain}
\chapter*{Lời nói đầu}
\addcontentsline{toc}{chapter}{Lời nói đầu}
Trong cuộc các mạng công nghiệp lần thứ tư, các hệ thống nhúng và cơ sở sản xuất thông minh sẽ được kết nối với nhau để tạo ra sự hội tụ kỹ thuật số giữa công nghiệp, kinh doanh, chức năng và quy trình bên trong. Một trong những yếu tố cốt lõi của cách mạng công nghiệp lần thứ tư là Internet Vạn Vật - IoT. IoT là một khái niệm về một viễn cảnh mà các thiết bị ở đó được định danh và có thể chia sẻ, giao tiếp với nhau.
\par
Với số lượng thiết bị vô cùng lớn, để kết nối chúng với nhau thì cần sử dụng một mạng truyền thông không dây. Có rất nhiều công nghệ truyền thông không dây đã được đề xuất như mạng viễn thông, Bluetooth, WiFi, ZigBee,… nhưng chúng đều không đáp ứng được đầy đủ yêu cầu mà IoT đặt ra là truyền dữ liệu khoảng cách xa và tiết kiệm năng lượng. Do đó, nhiều công nghệ truyền thông không dây khác được đề xuất để áp dụng hệ thống IoT như Sigfox, NB-Fi, RPMA và LoRa.
\par 
Là một công nghệ mới, công nghệ LoRa cần có những kỹ sư trẻ, nhiệt huyết tìm hiểu, ứng dụng vào thực tiễn đời sống và phát triển. Trước yêu cầu đó, em quyết định chọn đề tài \textbf{“THIẾT KẾ, PHÁT TRIỂN GIAO THỨC ĐA TRUY NHẬP LORA, TÍCH HỢP VÀO MẠNG WSN GIÁM SÁT MÔI TRƯỜNG”} nhằm góp phần xây dựng được một mạng cảm biến không dây sử dụng công nghệ LoRa để thu thập dữ liệu về những tham số môi trường cần thiết và đáp ứng được các tiêu chí về năng lượng tiêu thụ, chi phí, khoảng cách truyền dữ liệu.
\par
Qua đây, em cũng xin chân thành cảm ơn thầy giáo \textbf{TS. TRẦN QUANG VINH} đã trực tiếp định hướng, tạo điều kiện cần thiết và tận tình hướng dẫn để em có thể hoàn thành đồ án của mình. Em cũng xin chân thành cảm ơn các thành viên trong phòng nghiên cứu Hệ thống mạng và các ứng dụng thông minh \textbf{SANSLAB} (Smart Applications and Network Systerm Laboratory) đã nhiệt tình hỗ trợ, giúp đỡ em trong quá trình nghiên cứu và thực hiện đồ án.
\par
Trong quá trình làm đồ án, do kiến thức của em còn nhiều hạn chế và hiểu biết chưa rộng nên đồ án không tránh khỏi thếu sót. Em rất mong nhận được sự chị bảo và nhận xét quý báu của các thầy cô.
\par 
Em xin cam đoan các kết quả được trình bày trong đồ án là công trình nghiên cứu của em dưới sự hướng dẫn của giáo viên hướng dẫn. Các số liệu, kết quả trong đồ án là hoàn toàn trung thực, chưa được công bố trong bất kỳ công trình nào trước đây. Các kết quả được dùng để tham khảo đều được trích dẫn đầy đủ và theo đúng quy định.
\par 
Em xin chân thành cảm ơn!
\begin{tabbing}
 \hspace{8cm}\=\kill
   \> Hà Nội, ngày 30 tháng 5 năm 2018 \\ 
   \>  \hspace{2.5cm}    Tác giả\\ 
   \>   \hspace{1.5cm}  Hoàng Minh Mạnh		
 \end{tabbing} 