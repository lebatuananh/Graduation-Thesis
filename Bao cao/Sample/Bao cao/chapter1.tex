\chapter{Đặt vấn đề}
\label{chapter1}
Trong chương \ref{chapter1}, em xin trình bày khái quát về đề tài, mục tiêu mà đề tài hướng đến cũng như các phương pháp nghiên cứu được sử dụng để hoàn thành được đề tài ~\cite{Bankov2017}.
\section{Tổng quan về đề tài}
Đề tài “THIẾT KẾ, PHÁT TRIỂN GIAO THỨC ĐA TRUY NHẬP LORA TÍCH HỢP VÀO MẠNG WSN GIÁM SÁT MÔI TRƯỜNG” là đề tài nghiên cứu và tích hợp công nghệ LoRa vào thiết bị IoT. Trong cuộc cách mạng công nghiệp 4.0, IoT là một trong các xu thế, nó không chỉ được áp dụng trong các mô hình sản xuất mà còn được ứng dụng để phục vụ cuộc sống hằng ngày con người như nhà ở thông mình (Smart Home), thành phố thông minh (Smart City), giao thông thông minh (Smart Transport),…
\par
LoRa (Long Range) là kỹ thuật điều chế dựa trên kỹ thuật trải phổ, có phạm vi hoạt động lớn hơn rất nhiều so với các kỹ thuật cạnh tranh khác. Công nghệ truyền thông không dây LoRa được phát triển bởi Cycleo SAS, sau đó được mua lại bởi Semtech. Theo tài liệu \cite{1}, công nghệ LoRa đã được xuất hiện ở nhiều quốc gia trên thế giới. Ở châu Âu, nhiều dự án được triển khai trên toàn lãnh thổ hoặc được thí điểm trên khu vực rộng lớn. \textls[-5]{Một số dự án ở châu Âu như:}
\begin{itemize}
\item	KPN (Hà Lan), nó là dự án đầu tiên hoàn thành việc sử dụng mạng lưới LoRa (phủ sóng cả nước) cho thiết bị IoT,
\item	Orange (Pháp), trong quý đầu năm 2016, dự án đã bắt đầu triển khai ở 18 khu vực đô thị và được kỳ vọng sẽ triển khai trên khắp cả nước,
\item	Unidata, một công ty viễn thông của Ý đã ra mắt mạng LoRaWAN tại Rome và sớm sẽ triển khai trên toàn quốc.
\end{itemize}
Tại Mỹ, hai thành phố Philadelphia và San Francisco đã được chọn cho việc thử nghiệm các ứng dụng IoT như theo dõi tài sản và giám sát môi trường. Vào ngày 20/12 năm 2016, Semtech đã triển khai một mạng LoRaWAN tại New Zealand và hứa rằng một nửa dân số New Zealand có thể được sử dụng trong một tháng. Ở các khu vực khác trên thế giới như châu Á - Thái Bình Dương và châu Phi, công nghệ LoRa cũng bắt đầu được ứng dụng. Tại Việt Nam, công nghệ LoRa đang trong quá trình nghiên cứu để có thể áp dụng vào các dự án thực tế.
\par
Trong mạng cảm biến không dây WSN (Wireless Sensor Network), các nút trong mạng sẽ có nhiệm vụ thu thập dữ liệu môi trường xung quanh qua các cảm biến. Những dữ liệu đó sẽ được gửi đến Gateway qua các phương pháp truyền thông không dây. Sau đó, chúng được gửi lên máy chủ để tổng hợp, phân tích và xử lý. Từ những dữ liệu đã được xử lý này, con người sẽ đưa ra các biện pháp để ứng phó hoặc chính những nút này cũng có thể đưa ra các biện pháp.
\section{Mục tiêu của đề tài}
Mục tiêu của đề tài này là thiết kế, phát triển giao thức đa truy nhập cho mạng sử dụng module truyền thông LoRa và tích hợp giao thức truy nhập vào thiết bị BKRES, là các thiêt bị có kích thước nhỏ, yêu cầu tiêu thụ năng lượng thấp. Dựa vào những mục tiêu trên, những yêu cầu cụ thể được đưa ra như sau:
\begin{itemize}
\item Truyền dữ liệu 1 – 1 giữa nút và Gateway,
\item Truyền dữ liệu từ nhiều nút đến Gateway (n-1),
\item Các nút có khả năng tự động tham gia mạng,
\item Truyền nhận dữ liệu chính xác, ổn định,
\item Tìm ra được các nguyên nhân gây mất dữ liệu và khắc phục,
\item Tích hợp thuật toán đa truy nhập vào thiết bị IoT (BKRES).
\end{itemize}
\section{Các phương pháp sử dụng trong nghiên cứu và thiết kế}
Các phương pháp mà em sử dụng gồm có:
\begin{itemize}
\item   Tham khảo tài liệu: tham khảo tài liệu từ sách báo về điện tử, từ mạng internet, kế thừa và phát triển các kết quả nghiên cứu đã có,
\item   Quan sát, học hỏi: thảo luận cùng thầy và các bạn để đưa ra được hướng đi tốt nhất, đạt hiệu quả cao,
\item   Thực hành và sửa lỗi: xây dựng thuật toán thành các phần sau đó thí nghiệm nhiều lần và tìm hiểu, thảo luận để đạt được kết quả tối ưu nhất.
\end{itemize}\par
\section{Kết luận}
Mục tiêu của đề tài này là thiết kế, phát triển giao thức đa truy nhập cho mạng sử dụng module truyền thông LoRa và tích hợp giao thức đa truy nhập vào thiết bị BKRES. Để hoàn thành đề tài cần có  những kiến thức về công nghệ LoRa, đa truy nhập trong mạng không dây cùng với những hiểu biết về hệ thống BKRES. Những kiến thức này sẽ được trình bày trong Chương \ref{chapter2}.

